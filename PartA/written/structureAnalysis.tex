\documentclass[12pt,oneside,a4paper]{article}

\usepackage[a4paper,margin=1in]{geometry}

\begin{document}

\title{Design and Model Analysis}
\date{\today}
\author{Robert Holt -- 388648}

\maketitle

In general the model reflects the structure of the scenario that it
attempts to simulate, with relative accuracy balanced with simplicity;
every class corresponds either to physically relevant component in the
system, is a strategy, or is the mail generator -- which simulates an
exteral input. Small critiques of the model are discussed below.

\paragraph{Building Central to Domain Model}\mbox{}\\
The concept of a building is central to the simulation's domain model. The
simulation API currently accepts parameters specifying building but gives
the building no object instance. Rather than this, a Building class -- along
with a BuildingFactory -- might be appropriate, so that:
\begin{itemize}
  \item
  Rather than be concerned about integer arguments specifying
  building parameters, the Simualtion can simply accept a
  building to simulate mail with
  \item
  Rather than store a large number of integer constants at the
  top level (under Simulation), they could be stored in
  relevant building subclasses to be generated by the Factory
  \item 
  Supporting additional commandline arguments is easier to
  both add and read with a factory
\end{itemize}

\paragraph{Real World Concurrency}\mbox{}\\
The simulation models concurrency as interleaved steps between the
MailSorter and the DeliveryBots. Concurrent mail delivery in reality
involves independent agents. So where the simulation limits the actions
of a DeliveryBot based on the MailSorter, a real delivery bot might be
free to deliver mail while the mailsorter is stopped.

Since the simulation measures a DeliveryBot's effectiveness by the number of
steps it takes, and the number of steps is based on the MailSorter, the
simulation will artificially favor certain strategies and overlook others
that might work better in a real office building.

\paragraph{Logging}\mbox{}\\
The logging and printing logic is dispersed throughout the top level
Simulation method that governs the entire simulation. An improvement
would be to add a Logger that collects information as the simulation
runs and prints it at the end, so that the program is more extensible,
the Logger and Simulation are agnostic of one another and the logic is
separate.

\end{document}
